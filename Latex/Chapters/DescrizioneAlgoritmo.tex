\chapter{DescrizioneAlgoritmo}


\maketitle QuickSort

Quicksort è un algoritmo di ordinamento sequenziale ampiamente ritenuto essere l'algoritmo di ordinamento sequenziale più veloce per un ampio set di input, infatti il termine che tradotto letteralmente in italiano indica ordinamento rapido. È l'algoritmo di ordinamento che ha, nel caso medio, prestazioni migliori tra quelli basati su confronto. È stato ideato da Charles Antony Richard Hoare nel 1961.\\ 
È un algoritmo ricorsivo che usa il metodo "Divide and Conquer" per ordinare tutti i valori. Quicksort dal momento che scompone ricorsivamente i dati da processare in sottoprocessi, tale procedura viene chiamata Partition, preleva prima un pivot da una struttura dati, trova la sua posizione nell'elenco in cui deve essere posizionata. \\
Avremo due possibilità:
\begin{itemize}
\item I valori minori del pivot saranno posizionati nella parte sinistra dell'array
\item I valori maggiori o uguali saranno posizionati nella parte di destra dell'array
\end{itemize}




\maketitle Bitonic Sort



\end{document}