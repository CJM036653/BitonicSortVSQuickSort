\chapter{Conclusioni}
In seguito alle analisi effettuate, si può concludere abbastanza facilmente che, almeno per quanto riguarda le taglie di input considerate e un numero di processori non superiore a 32, Quick Sort sia più veloce di Bitonic Sort. Bisognerebbe comunque effettuare alcuni test aggiuntivi, in modo da verificare l'andamento anche oltre alle condizioni imposte dal calcolatore.
\'E possibile, inoltre, che il codice possa venire ottimizzato ulteriormente, specialmente nelle fasi di sincronizzazione di Quick Sort. Queste, infatti, comportano un overhead significativo all'aumentare del numero di processori. Di conseguenza, il numero massimo di processori utili a Quick Sort è 8, mentre Bitonic Sort potrebbe, potenzialmente, migliorare anche con più di 32 processori. Non è escluso, proprio per questo, che per numeri così elevati di processori Bitonic Sort ottenga prestazioni migliori, poiché la sua scalabilità è notevolmente superiore.