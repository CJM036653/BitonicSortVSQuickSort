\chapter{Introduzione}

L'ordinamento è uno dei problemi cardine di innumerevoli applicazioni scientifiche e ingegneristiche e le prestazioni degli algoritmi che lo risolvono sono di diretto interesse pratico.

Molti algoritmi sequenziali richiedono un tempo $O(NlogN)$ per eseguire l'ordinamento. Per aumentare la velocità di queste operazioni sono stati ideati alcuni metodi paralleli che sfruttano la presenza di più processori che possono svolgere operazioni in contemporanea e comunicare tra di loro.

In questa relazione verranno analizzate e messe a confronto le prestazioni di due algoritmi parallelizzati: il \textit{Bitonic Sort} e il \textit{Quick Sort}. Le due implementazioni includono le ottimizzazioni proposte, rispettivamente, in \cite{PaperBitonic} e \cite{PaperQuickSort}.
